\documentclass[12pt]{article}
\usepackage{amsmath}
\usepackage{amssymb}
\usepackage{amsfonts}
\usepackage[polish]{babel}
\usepackage[T1]{fontenc}
\usepackage[dvips]{graphicx}
\usepackage[cp1250]{inputenc}
\usepackage{caption}
\usepackage{enumerate}


\captionsetup[table]{name=Tabela}


\textheight 23.2 cm

\textwidth 6.0 in

\hoffset = -0.5 in

\voffset = -2.4 cm

\hyphenation{me-to-dy la-bo-ra-to-rium}

\begin{document}

%hę?
%\thispagestyle{empty}

\vspace*{3ex}
\begin{flushright}
{\large 23 stycznia 2020}
\end{flushright}

\begin{flushleft}
{\large Jakub Kujawa\\
Miko\l{}aj Kowalski\\
Grupa G8}
\end{flushleft}

\hskip3cm

\begin{center}

\Large {\bf Obliczanie ca\l{}ek $\int_{a}^{b}f\left( x\right)dx$ metod\k{a} Romberga.} \vskip2ex

{\large Projekt nr 2}

\end{center}

\vskip20ex



\section{Opis metody}
Do obliczania ca\l{}ek $\int_{a}^{b}f\left( x\right)dx$ zastosujemy metod\k{e} opracowan\k{a} w 1955 roku przez Wernera Romberga, kt\'ora opiera si\k{e} na wykorzystaniu z\l{}o\.zonego wzoru trapez\'ow:
\begin{equation}
\int_{a}^{b}f\left( x\right)dx \approx h_n \sum_{i=0}^n " f(a+ih) \label{Trapezy}
\end{equation}
gdzie n jest liczb\k{a} przedzia\l{}\'ow o r\'ownej d\l{}ugo\'sci, na kt\'ore dzielimy odcinek $[a,b]$ oraz
\[
h_n=\frac{b-a}{n}
\]
Symbol
\[
\sum "
\]
oznacza sum\k{e}, kt\'orej skrajne sk\l{}adniki s\k{a} dzielone na 2. 
\\
W naszym przypadku odcinek $[a,b]$ b\k{e}dzie dzielony na $2^{n}$ przedzia\l{}\'ow o r\'ownej d\l{}ugo\'sci. Zatem do wzoru (\ref{Trapezy}) b\k{e}dziemy podstawiali
\[
h_n=\frac{b-a}{2^n}
\]
Niech 
\begin{equation}
R(n,0)=h_n \sum_{i=0}^{2^n} " f(a+ih) \label{Romberg1}
\end{equation}
Najprostsze wyra\.zenie, od kt\'orego zaczniemy, to 
\[
R(0,0)=\frac{1}{2}(b-a)[f(a)+f(b)]
\]
Aby unikn\k{a}\'c wielokrotnego obliczania warto\'sci funkcji f w tych samych punktach, skorzystamy z rekurencyjnego wyznaczania
\begin{equation}
R(n,0)=\frac{1}{2}R(n-1,0)+h_n \sum_{i=1}^{2^{n-1}} f(a+(2i-1)h_n) \label{rekurencja}
\end{equation}
poniewa\.z w $R(n,0)$ wyst\k{e}puj\k{a} warto\'sci potrzebne do wyliczenia $R(n-1,0)$. Warto\'sci funkcji f w tych punktach musimy podzieli\'c na 2 oraz doda\'c do nich warto\'sci po\'srednie $a+h_n , a+3h_n, a+5h_n$ itd, co mo\.zna odczyta\'c we wzorze (\ref{rekurencja}). 
\\
Przybli\.zenia $R(n,0)$ tworz\k{a} pierwsz\k{a} kolumn\k{e} przybli\.ze\'n ca\l{}ki. Kolejne kolumny wyliczymy stosuj\k{a}c poni\.zszy wz\'or dla $m>0$, kt\'ory wynika ze wzoru Eulera-Maclaurina oraz ekstrapolacji Richardsona:
\begin{equation}
R(n,m)=R(n,m-1)+\frac{1}{4^m-1}[R(n,m-1)-R(n-1,m-1)] \label{dalsze_przyblizenia}
\end{equation}
Wobec tego dzi\k{e}ki wzorom (\ref{Romberg1}), (\ref{rekurencja}), (\ref{dalsze_przyblizenia}) znajdziemy wszystkie elementy tablicy tr\'ojk\k{a}tnej:
$$
\begin{array}{cccc}
R(0,0)\\
R(1,0) & R(1,1) \\
\vdots & \vdots & \ddots \\
R(n,0) & R(n,1) & \dots & R(n,n) 
\end{array}
$$
kt\'ora zawiera coraz lepsze przybli\.zenia szukanej warto\'sci ca\l{}ki. Najlepsze przybli\.zenie, to $R(n,n)$. Warto\'sci z tablicy tr\'ojk\k{a}tnej b\k{e}d\k{a} obliczane kolejno wierszami, zaczynaj\k{a}c od pierwszego.
\\
Obliczaj\k{a}c kolejne przybli\.zenia b\k{e}dziemy bada\'c b\l{}\k{e}dy wzgl\k{e}dem funkcji $integral(f,a,b)$ wbudowanej w program Matlab. Funkcja ta wyznacza warto\'s\'c ca\l{}ki z funkcji $f$ na przedziale $[a,b]$. Na podstawie wyznaczonych b\l{}\k{e}d\'ow wzgl\k{e}dnych oraz bezwzgl\k{e}dnych zostanie przeprowadzona analiza metody. 
\\
\vskip20pt

\section{Opis programu obliczeniowego}

W czasie tworzenia programu obliczeniowego zosta\l{}y stworzone nast\k{e}puj\k{a}ce funkcje:
\begin{enumerate}
\item $Romberg(f,a,b,M)=[x,R]$:
\\
przyjmuje za argumenty funkcj\k{e} $f$, ko\'nce przedzia\l{}u $[a,b]$, na kt\'orym wyznaczamy warto\'s\'c ca\l{}ki, oraz liczb\k{e} M, kt\'ora okre\'sla na ile odcink\'ow b\k{e}dzie dzielony przedzia\l{} $[a,b]$ (na $2^M$ r\'ownych odcink\'ow). Wyznacza warto\'s\'c $\int_{a}^{b}f\left( x\right)dx$ metod\k{a} Romberga oraz tablic\k{e} R, w kt\'orej umieszczone s\k{a} kolejnych przybli\.zenia warto\'sci badanej ca\l{}ki.

\item $testRomberg(f,a,b,n,s)$:
\\
przyjmuje za argumenty funkcj\k{e} $f$, ko\'nce przedzia\l{}u $[a,b]$, na kt\'orym wyznaczamy warto\'s\'c ca\l{}ki, liczby ca\l{}kowite n oraz s. Testuje metod\k{e} Romberga dla warto\'sci M od 1 do n z krokiem co s. Dla ka\.zdego M wypisuje tabel\k{e} w formacie LaTex i tworzy wykresy b\l{}\k{e}d\'ow wzgl\k{e}dnych i bezwgl\k{e}dnych. 
 
\end{enumerate}

\vskip20pt

\section{Przyk\l ady obliczeniowe}

\noindent 
By zbada\'c poprawno\'s\'c naszego programu sprawdzimy jego dzia\l{}anie dla pewnych konkretnych funkcji:
\begin{enumerate}
\item $f(x)=1+sin(1/x)$
\item $f(x)=sin(x)$
\item $f(x)=tg(x)$
\item $f(x)=x^{-1}$
\item $f(x)=2^x$
\item $f(x)=x^3+x^2+x+1$

\end{enumerate}

\vskip20pt

\section{Analiza wynik\'ow}

\end{document}